\chapter{सोप्या मार्गाचे शहाणपण}

काल दुपारची गोष्ट. माझ्या मोबाईलचं वाय-फाय काही चालत नव्हतं. खूप प्रयत्न केले, सेटिंग्ज तपासली, इंटरनेटवर माहिती वाचून काही गोष्टी केल्या, युट्युबवरील व्हिडीओ, एक ना अनेक उपाय केले पण ते वाय-फाय काही प्रतिसाद देईना. संगणक क्षेत्रातील असूनही उत्तर सापडत नसल्याने साहजिकच चीड-चीड वाढली. मोबाईलला, त्यातल्या सॉफ्टवेअरला, एवढच काय, संपूर्ण तंत्रज्ञान क्षेत्राला आणि नेहमीच्या सवयीने शासनाला (!) पण नावं  ठेऊन झाली पण समस्या काही सुटेना. संध्याकाळी घरी आल्यावर माझ्या छोट्या मुलीला हे कळताच ती म्हणाली की बाबा जरा डिस्कनेक्ट करून पुन्हा कनेक्ट करून बघा ना? आणि खरोखरच त्याने ते अडेल-तट्टू वाय-फाय चालू झाल ना! उगाच काहीतरी क्लिष्ट, गुंतागुंतीचे, किचकट करत बसलो पण उत्तर मात्र सोपे होते. ह्यालाच म्हणतात ‘काखेत कळसा गावाला वळसा’. या मेंटल मॉडेलला (मन:प्रारूप) अथवा विचार चित्राला ‘ऑकम्स रेझर’ किंवा साध्या भाषेत ‘सोप्या मार्गाचे शहाणपण’ म्हणू शकतो. २०२३ च्या “भारतीय दूरसंचार विनियामक प्राधिकरण” (टी-आर-ए-आय)  यांच्या अहवालानुसार, भारतातील ८५% पेक्षा जास्त ब्रॉडबँड तक्रारी केवळ मोडेम रीस्टार्ट करून किंवा सैल कनेक्शन तपासून सोडवल्या जातात तरीही बहुतेक लोकं, अधिक जटिल अशा तांत्रिक बिघाडांचा अंदाज लावतात आणि साध्या उपायांकडे दुर्लक्ष करतात. नेहमी खोल विचार करण्याची व गुंतागुंतीची स्पष्टीकरणे देण्याची ही प्रवृत्ती खरंतर एक मानसिक पूर्वग्रह दर्शवते. येथे ‘सोप्या मार्गाचे शहाणपण’ हे विचारचित्र आपल्याला सुधारण्यास मदत करते.

१४व्या शतकातील इंग्लिश तत्त्वज्ञ ‘विल्यम ऑफ ऑकम’ यांच्या नावावरून हे विचारचित्र आले आहे. "रेझर" म्हणजे दाढी करण्याचा ब्लेड. आवश्यक  नसलेली गृहितकं (ऍझमशन्स) दूर करायची, ही यामागची कल्पना आहे. ते आपल्याला सांगते की तुमच्याकडे एखाद्या समस्येसाठी अनेक उत्तरे-पर्याय असतील तर त्यातले असे निवडा की ज्यात सर्वात कमी गृहीतके आहेत किंवा अधिक सोप्या शब्दांत म्हणायचे झाले तर, सर्वात सोपा उपाय बहुधा योग्य असतो. ‘ऑकमचा रेझर’ हे सोपे उत्तर नेहमीच बरोबर असेल याची हमी देत नाही, परंतु ते असे सुचवते की सर्वात कमी गुंतागुंतीच्या स्पष्टीकरणापासून सुरुवात करणे हा समस्येच्या समाधानाकडे जाण्याचा सर्वात कार्यक्षम मार्ग असतो. या जगात जिथे माहिती विपुल आहे आणि जटिलतेचे वैभव दाखवले जाते, तिथे सोपेपणा केवळ सुंदरच नाही तर तो धोरणात्मक सुद्धा असू शकतो. या विचारचित्राची काही उदाहरणे पाहुयात. 

प्रसिद्ध उद्योगपती रतन टाटांना मध्यमवर्गासाठी कार बनवायची होती. सर्वात महत्वाची गोष्ट म्हणजे ती किफायती पाहिजे, त्याकाळात म्हटले तर १ लाख रुपयात. त्यांच्या संशोधक टीमने कार मध्ये आत्यंतिक गरजेचे काय असते तेच ठेवून, इतर कमी महत्वाच्या गोष्टींना फाटा देऊन, इतर किफायतशीर बदल करून, तसे कार मॉडेल बनवून दाखवले आणि ते म्हणजे ‘नॅनो’. 

गूगल कंपनीच्या आधी सुद्धा अनेक प्रसिद्ध शोध-आंतरजाल-स्थळे होती. ती माहितीने गजबजलेली वाटायची. अतिशय नाविन्यपूर्ण आणि कार्यक्षम शोध-प्रणाली बरोबरच गुगलने एक महत्वाचा बदल आणला तो म्हणजे, अतिशय सोप्पे शोध-संकेत-स्थळ (सर्च वेब पेज). टाईप करण्यासाठी एक बॉक्स आणि ‘शोध घे’ असे सांगण्यासाठी १-२ बटन्स, बस्स!! बाकी सगळे कोरे. या संकेत स्थळाच्या जोरावर गुगलने केवढी मोठी भरारी मारलीये. 

आर्थिक व्यवहारांसाठी आपण वापरात असलेल्या भीम-पे अथवा गुगल-पे च्या मागे असलेल्या यु-पी-आय (युनिफाइड पेमेंट्स इंटरफेस) ची जबरदस्त वाढ ओकॅमच्या रेझरचे उदाहरण आहे. आधीच्य ऑनलाईन पेमेंट सिस्टममध्ये अकाउंट नंबर, बँकेचा आय-एफ-एस-सी नंबर इत्यादी बाबी आवश्यक होत्या. पुन्हा प्रत्येक बँकेची संकेत-स्थळे वेगळी. यु-पी-आयने ही सर्व क्लिष्टता काढून एका साध्या क्यू-आर कोड स्कॅनपर्यंत काम कमी केले, अनावश्यक गुंतागुंत दूर केली आणि आता ते दरमहा १० अब्जाहून जास्त व्यवहार करत आहे.

गुंतवणूक क्षेत्रातही बऱ्याच क्लिष्ट रणनीती वापरल्या जातात. त्यातल्या तज्ज्ञांचं ठीक आहे पण सामान्य गुंतवणूकदारही अनेकदा बाजारपेठेतील किमतीच्या चढ-उताराची अचूक वेळ (मार्केट टायमिंग) साधणे, क्लिष्ट सूत्रे-गणितीय प्रणाल्यांसारखी साधने निवडणे किंवा जटिल ट्रेडिंग अल्गोरिदम याचा वापर करताना दिसतात. गरज नसताना, समजत नसतानाही अनावश्यकपणे गुंतागुंतीच्या मार्गावर जातात. खरंतर, बऱ्याच वेळेला असे दिसून येते की साधे इंडेक्स-फंड सुद्धा या जटिल रणनीतींपेक्षा चांगले प्रदर्शन करतात, तेही अल्पशा खर्चात.

आपण उत्पादनांमध्ये अनेक व कदाचित अनावश्य वैशिष्ट्ये (फीचर्स) भरतो, संस्थांमध्ये अति स्तर आणि रिपोर्टींगची गुंतागुंत करतो आणि अनेकदा असे मानतो की अधिक क्लिष्ट म्हणजे फार भारी. खरे सौंदर्य सोपेपणात आणि साधेपणात आहे. मान्य आहे की साधेपणाचाही अतिरेक नको पण अतिआवश्यक ते तर ठेवायलाच पाहिजे. कमीतकमी गोष्टीत जास्तीत जास्त परिणामकारकता कशी आणता येईल हा मुख्य विचार आहे. 

‘ऑकम्स रेझर’ची फक्त उपयुक्तता समस्या-सोडवण्यासाठी आहे असे नाही. ती जीवनाबद्दलची एक तात्त्विक भूमिका आहे. एका अशा जगात जे आपल्याला सातत्याने अधिक गुंतागुंतीकडे ढकलत आहे, तिथे साधेपणा निवडणे हा एक महत्वाचा पर्याय बनतो. 

