\chapter{जुनं ते सोनं}

सध्याच्या काळात पैसे गुंतवण्यासाठी अनेक पर्याय उपलब्ध आहेत. बँकेत मुदत-ठेव (फिक्स्ड डिपॉझिट), शेअर बाजार, म्युच्युअल फंड्स, जमीन-घर खरेदी आणि सर्वात आधुनिक मार्ग म्हणजे क्रिप्टो चलन. पण त्यातील एका पर्यायाची लोकप्रियता अगदी अनादी काळापासून जशीच्या तशी आहे, ते म्हणजे, सोनं. मागणी-पुरवठ्याचे नियम त्यालाही लागत असले तरी सर्वसाधारणपणे भाववाढीच्या बरोबर कमी-जास्त होणारे, सुरक्षित, जागतिक स्तरावर विश्वासार्ह आणि तरल (लिक्वीड) म्हणजेच वेळप्रसंगी कधीही मोडून पैसे उभे करता येणारे, ते म्हणजे सोनं. म्हणूच एका पिढीकडून दुसरीकडे, लग्नप्रसंगी एका कुटुंबातून दुसऱ्याकडे देवाणघेवाण होऊन, पै-पै साठवून हे चमकदार ‘धन’ सांभाळले जाते आणि सणासुदीला ‘लक्ष्मी’ स्वरूपात पुजले पण जाते. शेकडो वर्षे झाली पण यात काही बदल नाही. अनेक राजे-राजवाडे होऊन गेले, अनेक तेजी-मंदीचे काळ येऊन गेले पण सोन्याची शान तशीच्या तशीच आहे. अशाप्रकारे चिरकाल जिवंत राहणाऱ्या कल्पनांच्या ‘मेंटल मॉडेल’ (मन:प्रारूप) अथवा विचार-चित्राला ‘लिंडी इफेक्ट’ म्हणजेच सोप्या भाषेत ‘जुनं ते सोनं ‘ असे म्हणतात. जे आतापर्यंत, हजारो वर्षे टिकले आहे ते यापुढेही अनंतकाल टिकेल अशी या विचारचित्रामागची भूमिका आहे. 
१९६० च्या दशकात अमेरिकेतील विचारवंतांनी ब्रॉडवे नाटकांच्या दीर्घकालीन अस्तित्वाचे निरीक्षण करत "लिंडी इफेक्ट" चा सिद्धांत मांडला. नंतर, नसीम निकोलस तालेब यांनी त्याला लोकप्रिय केलं. या सिद्धांतानुसार, नष्ट न होणाऱ्या गोष्टी, जसे की कल्पना, तंत्रज्ञान, पुस्तके किंवा परंपरा या जितक्या जुन्या असतात, तितके त्यांचे भविष्य अधिक उज्ज्वल व टिकाऊ असते. साधे उदाहरण सांगायचं तर, जर एखादं पुस्तक ५० वर्षं अस्तित्वात राहिलं असेल, तर ते अजून ५० वर्षं टिकण्याची शक्यता जास्त आहे. शेकडो वर्षांपूर्वी चाणक्य यांनी लिहिलेलं अर्थशास्त्र. त्याच्यातील राज्यकारभार आणि धोरणातील धडे आजही एमबीए वर्गात आणि राजनीतीच्या-धोरणात्मक प्रशिक्षणात गिरवले जातात. का? कारण कालानुरूप असल्याने प्रत्येक पिढीत ते टिकून राहिले आहेत आणि या विचारचित्रानुसार ते भविष्यातही राहतील हा कयास आहे. अशीच काही इतर उदाहरणे पाहुयात. 
भारतातील कोणत्याही शास्त्र-अभियांत्रिकीच्या विध्यार्थ्याला विचारलं की १२वी ला भौतिक-शास्त्राचा अभ्यास कशातून केला तर बहुतेकांचं उत्तर येईल एच.सी. वर्मांचं "कॉन्सेप्ट्स ऑफ फिजिक्स". १९९० च्या दशकातील हे पुस्तक अजूनही टिकून आहे. अशीच प्रत्येक शाखेची, त्यातील विषयांची जुनी-लोकप्रिय पाठ्यपुस्तके (क्लासिक्स) असतात त्यावाचून अभ्यासाचे ‘पान’ हालत नाही. काळानुसार काही बदल होतात पण बहुतांशी मूळ गाभा तोच राहतो. इतर समाज-जीवनातही आजही आपण संत ज्ञानोबारायांचे-तुकोबारायांचे अभंग, दासबोधातील समर्थ रामदास स्वामींच्या ओव्या सहज उद्धृत करीत असतो कारण ते अजून समयोचित (रिलेव्हन्ट) राहिले आहेत. शेवटी, जुनं ते सोनं. 
फॅशन प्रमाणेच आहार विषयात कायम काहीतरी नवीन वारे (ट्रेंड्स) वाहत असतात. कधी किटो, तर कधी लो-कार्ब्स, कधी सतत खा, तर कधी दोनच वेळेला. असे एक ना अनेक. भूक लागेल तेंव्हा, थोडा अवकाश (म्हणजे पोटात जागा) ठेवून, आणि आपली आजी जे खात होती ते साधारणतः आपणही खाल्लं तर ते चांगलं मानवतं, नाही का? हेच शाश्वत तत्व आहे. या क्षेत्रात अजून एक ट्रेंड दिसुन येतो. अमेरिकेतील (आणि आता भारतातही) प्रसिद्ध असलेली कॅफे मध्ये काही पेय (आणि त्याच्या किमती) पहिल्या की तोंडात बोटे (आश्चर्याने!!) जातात. ‘टर्मरिक लॅट्टे’ हा प्रकार भारी लोकप्रिय झाला आहे. खरं म्हणजे, भारतीय घरात वर्षानुवर्षे प्यायले जाणारे ते हळदीचे दूध!!! नवीन उत्पादन म्हणून समोर आणलं आणि प्रसिद्ध झालं शेकडो वर्षे ज्ञात असलेल्या त्याच्या गुणधर्मानेच. 
कितीही ई-कॉमर्स द्वारे व्यवहार होत असले तरी, जवळच्या किराणामालाच्या दुकानात असलेली आपली वही-खातं  अजून चालू आहे. लहान कुटुंबाद्वारे चालवलेली, वेळेला उसने देणारी ही व्यवस्था अजून टिकून आहे, कारण संबंध, स्थानमाहात्म्य आणि विश्वास हे कायम टिकणारे असतात.
आरोग्यक्षेत्रात कितीही प्रगती झाली, नवनवीन तंत्रज्ञान आले, दूरस्थ (रिमोट) पद्धतीने उपचार चालू झाले तरी अनुभव सिद्ध डॉक्टरांच्या रोग-निदानाला, नाडी परीक्षेला अजूनही स्थान आहे. त्यांच्या आश्वासक बोलण्यातच अर्धा आजार पळून जात असावा असे वाटते. त्यामुळे या कृत्रिम बुद्धिमतेच्या (‘आर्टिफिशिअल इंटेलिजन्स’), यंत्रमानवांच्या  जगातही (आणि भविष्यातही) डॉक्टरांचे महत्व अबाधितच राहणार आहे. 
हजारो वर्षांच्या इतिहासाने समृद्ध असलेल्या आपल्या देशात, जागोजागी, पदोपदी 'लिंडी इफेक्ट’ची उदाहरणे सापडतात. मंदिरे, शास्त्र, पाककला, परंपरा आणि जीवन मूल्ये आपण काहीप्रमाणात का होईना जपली आहेत. आपण अनेकदा (गरज-कारण नसताना) नवतेच्या मागे धावतो, काहीतरी पूर्वीपेक्षा वेगळं-हटके करायचं या हट्टापायी. वर्षातून दहा वेळा दिशा बदलणारा (‘पिव्हट’  करणारा) स्टार्टअप. भाषणात दुसऱ्या दिवशी धोरण बदलणारा पलटूराम राजकारणी. रोज नवे ब्रँडिंग करून सल्ला देणारा इन्फ्लुएन्सर, अशी एक ना अनेक उदाहरणे आहेत. 
‘लिंडी इफेक्ट’ हा काही नवीन गोष्टींना नाकारण्याचा सल्ला नक्कीच देत नाही. कोणतीही गोष्ट करताना, जे टिकले आहे त्याचाही विचार करा एवढेच तो सुचवतो. अनेक स्थित्यंतरातही जे शाबूत राहिलं त्याचा आपण खोल विचार करायला हवा. पुढच्या वेळी तुम्हाला निर्णय घ्यायचा असेल,  काय वाचावे, काय खावे, कोणत्या सल्ल्यावर विश्वास ठेवावा, तेंव्हा विचार करा की याने काळाची कसोटी पार केली आहे का?   

