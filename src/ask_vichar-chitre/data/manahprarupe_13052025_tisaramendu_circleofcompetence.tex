\chapter{गड्या, आपुला गाव बरा}

हिंदी चित्रपटसृष्टीचे महानायक अमिताभ बच्चन यांनी सत्तर-ऐंशीच्या दशकात एकामागून एक यशस्वी चित्रपट दिले. ते प्रसिद्धीच्या शिखरावर होते. १९८३ च्या ‘कुली ‘ चित्रपटातील दुर्दवी अपघातातून सावरून आल्यानंतर तर ते जनमानसाच्या गळ्यातील ताईत बनले होते. कदाचित याच लोकप्रियतेचा फायदा व्हावा म्हणून (किंवा खरोखरच्या जनसेवेच्या ओढीने) त्यांनी राजकारणात पाऊल टाकले. लोकसभेच्या निवडणुकीत अलाहाबाद (आताचे प्रयागराज) येथून त्यांनी माजी मुख्यमंत्री बहुगुणांचा मोठा पराभव केला. पण तीन-एक वर्षातच, झालेल्या आरोपांमुळे, त्यांनी वैतागून, नाराज होऊन, राजकारणाला ‘दलदलीसारखं’ म्हणत या क्षेत्राचा निरोप घेतला. जे व्यक्तिमत्त्व एका चित्रपटाला खांद्यावर उचलू शकत होतं, ज्याच्या संवादांनी लाखो लोक भारावून जायचे, आणि ज्याने एक पूर्ण युग घडवलं, तेच व्यक्तिमत्त्व राजकारणातील खेळी, सत्तेचे समीकरण आणि प्रशासन यांमध्ये अपयशी ठरलं. कारण तो त्यांचा ‘रंगमंच’ नव्हता. खरं तर हे अपयश नव्हे, तर एक शिकवण आहे. एका क्षेत्रातील टाळ्यांची ग्वाही दुसऱ्या क्षेत्रात यश मिळवून देईलच, असं नाही. शेवटी, ‘गड्या, आपुला गाव बरा’ असे म्हणावे लागते. 

आज बहु-गुणी असण्याचा गौरव होतो. अगदी ‘जॅक ऑफ ऑल ट्रेड्स’ म्हणजेच ‘सगळ्याच गोष्टी थोड्याफार प्रमाणात येतात’ ही गोष्ट काहीशी प्रतिष्ठेची मानली जाते.  येथे ‘सर्कल ऑफ कॉम्पिटन्स’ (क्षमता-वर्तुळ) या मेंटल मॉडेल (मन:प्रारूप) म्हणजेच विचारचित्राची गरज भासते. ते आपल्याला क्षमतेच्या मर्यादांची जाणीव करून देते. ते आपल्याला सांगते की तुम्हाला काय येतं ते ओळखा आणि त्याहून महत्त्वाचं म्हणजे, काय येत नाही तेही! जे आपल्याला नीट कळले आहे, येते आहे, त्यात राहावे व प्रगती करावी नाहीतर ‘एक ना धड आणि भाराभर चिंध्या’ होण्याची शक्यता असते. गुंतवणूकदार चार्ली मंगर यांनी या संकल्पनेला प्रसिद्धी मिळवून दिली. उदाहरणार्थ, प्रसिद्ध गुंतवणूकदार वॉरेन बफे (चार्ली मंगर यांचे सहकारी) यांनी अनेक दशकांपर्यंत तंत्रज्ञान कंपन्यांमध्ये गुंतवणूक केली नाही. त्यांचा तंत्रज्ञानाला विरोध नव्हता, पण त्यांना त्या कंपन्यांचे दीर्घकालीन मूल्य समजत नाही, हे त्यांनी प्रामाणिकपणे मान्य केलं. ही विनयशीलता काही नकारात्मकता नव्हे तर तीच त्यांना डॉट-कॉम क्रॅशपासून वाचवू शकली आणि बर्कशायर हाथवे सारखी यशस्वी कंपनी घडवता आली. अशा या महत्वाच्या विचारचित्राची अजून काही उदाहरणे पाहुयात. 

गुंतवणूक क्षेत्रात काही काळापूर्वी क्रिप्टो करंसीचे (आभासी-सांकेतिक चलन) वारे वाहत होते. बऱ्याच लोकांना त्याच्या मागची संकल्पना, त्याचे मूल्य जोखण्याची पद्धत असे काहीही माहिती नव्हते तरी काहीजण आपल्या मित्रांकडून ‘टिप्स’ घेऊन आपली बचत टोकनमध्ये गुंतवायचे, अगदी ‘कळपातील मेंढ्यांप्रमाणे’. अशा प्रसंगी, आपल्याला जरा माहिती असलेल्या म्युच्युअल फंड किंवा इंडेक्स फंडमध्ये राहणं कधीही जास्त शहाणपणाचे नाही का?

नोकरीत सुरवातीपासून कष्ट करून, ज्ञान सम्पादन करून, आपल्या तांत्रिक कौशल्यावर अनेक जण बढत्या मिळवतात. थोड्या वर्षांनी ‘सिनियर’ (वरिष्ठ)  झाल्यावर आपोआप व्यवस्थापकाची (‘मॅनेजर’) ची जबादारी येते (खरंतर, दिली जाते). त्याची काहीही आवड नसताना, क्षमता नसताना फक्त सामाजिक दबावामुळे आणि वेतन वाढीमुळे लोकं ते स्वीकारतात. पण सत्य थोड्याच दिवसात कळते, तेथील गोंधळ, वरच्यांकडून आणि खालच्यांकडूनही ऐकावी लागणारी बोलणी पाहून काहींना नक्की वाटते की ‘गड्या, आपुला गाव बरा’, परत आवडत्या आणि चांगले जमणाऱ्या तांत्रिक कामातच जावे आणि त्यातच उत्तरोत्तर प्रगती करावी. 

एखाद्याने ‘मधुमेहावर घरगुती उपाय’ असा व्हॅट्सऍप वर मेसेज पाठवला तर विचार करायला पाहिजे, की पाठवणाऱ्याचे ते अधिकार क्षेत्र आहे का? त्यात त्याचा अभ्यास किती आहे म्हणजेच हा विषय त्याचा ‘क्षमतेच्या वर्तुळातील’ आहे का? हे पाहूनच ते सल्ले  मानावे की नाही हे ठरवावे. खऱ्या डॉक्टरऐवजी अशा सल्ल्यावर विश्वास ठेवणं धोकादायक ठरू शकतं.

हातात कधी बॅट न घेतलेले, सचिनला कव्हर ड्राइव्ह कसा मारायचा ते सांगत असतात? पुलंच्या भाषेत सांगायचे झाले तर “आपण स्वतः पुणे महानगरपालिकेत, उंदीर मारायच्या विभागात आहोत नोकरीला, हे विसरून ‘अमेरिकेची आर्थिक घडी नीट बसवण्याचा खरा मार्ग कोणता?’ यावर  मत ठणकावतात”. या उदाहरणांत आपण ‘क्षमता-वर्तुळाच्या’ बाहेर जात आहोत हे कळले पाहिचे. याला उपाय म्हणजे जागरूकता, अभ्यास आणि विनयशीलता. 

आजच्या चढाओढीच्या आणि गलबल्याच्या जगात, ’क्षमता-वर्तुळ’ हे विचारचित्र आपल्याला स्वतःची एक शांत, पण खंबीर ओळख देते. एखाद्या गोष्टीचा निर्णय घेण्याआधी तिला नीट समजून घ्या आणि आपल्या क्षमतेत नसेल तर दुसऱ्या तज्ज्ञांची मदत घ्या, असे सुचवते. मग विचार येतो की आपण असेच क्षमता-वर्तुळात अडकून, संकुचित (कूपमंडूक), राहायचे का? तर, नाही.  “गड्या, आपुला गाव बरा” म्हणताना  हे विचारचित्र आपल्याला नवीन शिकण्यापासून थांबवत नाही. तुम्ही हे वर्तुळ मोठे करू शकता, ‘गावाचा’ ‘देश’ करू शकता तर ‘देशा’चे ‘जग’. म्हणजेच तुमच्या क्षमतेचं क्षेत्र तेच ठेवून त्यात उत्तरोत्तर प्रगती करणे, हा खरा उद्देश आहे. 

