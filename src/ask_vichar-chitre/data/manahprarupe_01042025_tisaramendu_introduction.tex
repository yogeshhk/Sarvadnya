\chapter{‘विचार-चित्रांची’तोंडओळख}

जगातील सर्वात प्रसिद्ध गुंतवणूकदार कोण? असे विचारल्यावर बहुतेकांचे उत्तर असेल - वॉरन बफे. पण त्यांचे तुलनेने कमी प्रसिद्ध, तरीही असामान्य प्रतिभेचे सहकारी म्हणजे चार्ली मंगर. कोणत्याही समस्येवर किंवा प्रसंगावर चार्ली विविध दृष्टिकोनातून विचार करायचे. या दृष्टिकोनांना ते मेंटल मॉडेल्स (मन:प्रारूपे) म्हणत. सोप्या भाषेत याला आपण 'विचार-चित्रे' म्हणू शकतो, कारण एखाद्या परिस्थितीला दिला जाणारा प्रतिसाद हा आपल्या मनात कोरलेल्या विचार-चित्रावर अवलंबून असतो. मानवी मेंदू सुरवातीपासून  विचार करण्याची ऊर्जा वाचवण्यासाठी अशा विचार-चित्रांचा आधार घेतो आणि सहज निर्णय घेतो.

आपण स्वतःला तर्कशुद्ध आणि वस्तुनिष्ठ विचार करणारा प्राणी समजतो, पण प्रत्यक्षात असे सतत घडत नाही. अनेकदा भावनांच्या प्रभावाखाली, सवयीने किंवा पूर्वग्रहदूषित पद्धतीने निर्णय घेतले जातात. प्रदीर्घ अनुभव आणि वारंवारतेमुळे अशा पद्धती विकसित होतात. विविध प्रसंगांना तोंड देण्यासाठी वेगवेगळी विचार-चित्रे वापरण्यात येतात. आपण सर्वच जण यांचा वापर करीत असलो तरी काही विशिष्ठ विचारचित्रे अतिशय प्रभावीपणे उपयोगात आणता येतात. पुढील लेखांमध्ये अशीच काही विचार-चित्रे उदाहरणांसह पाहुयात.

या सदरचे-लेखमालेचे शीर्षक 'तिसरा मेंदू' असे आहे. हे बाह्य किंवा अतिरिक्त बुद्धिमत्तेचे प्रतीक आहे. जर आपल्या भात्यात अनेक मेंटल मॉडेल्स (विचार-चित्रे) असतील, तर योग्य प्रसंगात योग्य मॉडेल वापरणे हीच अतिरिक्त बुद्धिमत्ता ठरते. त्यामुळे आपली विचारक्षमता वाढते आणि प्रश्नांचा सर्वांगीण विचार करता येतो.

यशाच्या पायऱ्या चढताना जबाबदाऱ्या वाढतात. गतिमान आणि गुंतागुंतीच्या जगात घेतलेल्या निर्णयांचे परिणाम दूरगामी असतात. म्हणूनच यशस्वी लोक कळत-नकळत विचार-चित्रांचा उपयोग करतात. निर्णय घेण्यासाठी, संधी ओळखण्यासाठी आणि अचूक ज्ञान मिळवण्यासाठी ही पद्धती उपयुक्त ठरते. विचार करण्याची क्षमता सरावाने वाढवता येते. भविष्यातील संभाव्य परिस्थितींसाठी विविध विचार-चित्रे वापरून आढावा घेता येतो. नक्की काय महत्त्वाचे आहे? त्वरित आणि दूरगामी परिणाम काय असतील? यासाठी विचार-चित्रे मदत करतात. नियमित वापर केल्यास ही मॉडेल्स भात्यातील बाणांप्रमाणे सहज वापरता येतात. नेहमीच्या वापरातील, तुलनेने सोपे आणि प्रसिद्ध असे एक विचारचित्र आणि त्याची एक-दोन उदाहरणे पाहुयात. 

“स्वॉट विश्लेषण” 
या पद्धती मध्ये एखाद्या कल्पनेचे, प्रकल्पाचे, संस्थेचे किंवा चक्क माणसाचे सुद्धा मूल्यमापन करता येते. “स्वॉट” याच्या चार अद्याक्षरांच्या आधारे ते होते, म्हणजेच ‘एस’ (स्ट्रेंग्थ, सामर्थ्य), ‘डब्ल्यू’ (विकनेस, कमकुवत बाजू), ‘ओ’ (ओप्पोर्च्युनिटी, संधी) आणि ‘टी’ (थ्रेटस, धोके). 

उदाहरणार्थ एखाद्या विद्यार्थ्याचे “स्वॉट विश्लेषण” असे असू शकते की त्याचे ‘एस’ म्हणजे शास्त्र व गणित या विषयात चांगली गती असणे, त्याचे ‘डब्ल्यू’ म्हणजे संभाषण कौशल्याचा अभाव, ‘ओ’ म्हणजे विविध अभियांत्रिकी व तंत्रज्ञानाच्या संशोधन क्षेत्रात असलेल्या अमाप संधी, त्याचप्रमाणे ‘टी’ म्हणजे, त्याच संधींसाठी असलेली प्रचंड स्पर्धा. असे विश्लेषण वैयक्तिक प्रमाणेच, संस्थात्मक किंवा व्यावसायिक स्तरावरही करता येते. अजून एक उदाहरण पाहुयात. नवीन मोबाईल फोन बाजारात आणण्याआधी त्याचे केलेले स्वॉट विश्लेषण असे असू शकते  ‘एस’ (स्ट्रेंग्थ, सामर्थ्य), म्हणजे वाढवलेली संगणकीय क्षमता आणि कृत्रिम बुद्धिमत्तेचा (एआय) अंतर्भाव.  ‘डब्ल्यू’ (विकनेस, कमकुवत बाजू) म्हणजे वाढलेली किंमत आणि बॅटरीचा खप. ‘ओ’ (ओप्पोर्च्युनिटी, संधी)  म्हणजे नवीन तंत्रज्ञान आणि सुविधांसाठी वाढती मागणी, त्याचप्रमाणे  ‘टी’ (थ्रेटस, धोके) म्हणजे  वेगवान तांत्रिक बदल, सायबर सुरक्षेचे धोके आणि मोठ्या ब्रँड्सकडून तीव्र स्पर्धा. या पद्धतीने विचार केला की सर्व महत्त्वाच्या बाबींचा विचार झाला आहे याची खात्री होते.
विचार-चित्रांचा प्रभावी वापर
अशाप्रकारची अनेक विचार चित्रे आपण उपयोगात आणू शकतो. ती अतिशय प्रभावी असली तरी प्रत्येक समस्येला रामबाण उपाय म्हणून सरसकट वापरता येत नाही. कोठे कोणते ‘मेंटल-मॉडेल’ (विचार-चित्र) वापरायचे याचे तारतम्य बाळगावे लागते. अनुभवाने तेही जमायला लागते. कालांतरानी या तयार विचारचिंत्रांबरोबरच आपण स्वनिर्मित मॉडेल्स ची पण भर घालू शकतो. चार्ली मंगर यांच्या म्हणण्यानुसार त्यांच्याकडे १०० हून अधिक मेंटल मॉडेल्स होती, आणि ती विविध क्षेत्रांतून घेतलेली होती. इतक्या मोठ्या संख्येने विचार-चित्रे शिकणे अवघड असले, तरी सुरुवात करणे कठीण नाही! तर मग, तुम्ही विचार-चित्रे समजावून घेण्यास व वापरण्यास सुरुवात करताय का? बघा, काही फायदा होतोय का!
