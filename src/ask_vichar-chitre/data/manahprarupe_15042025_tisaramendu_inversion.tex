\chapter{उलटा विचार, यशाचा आधार}

एखादा माणूस सतत नकारात्मकतेने बघत असेल, तर ते चांगले मानले जात नाही. पण जर हा दृष्टिकोन समजून आणि योग्य प्रकारे वापरता आला, तर तो खूप फायद्याचा ठरू शकतो. हाच विचार एका प्रसिद्ध मेंटल मॉडेलचा (मन:प्रारूप) म्हणजेच ‘विचार-चित्राचा’ गाभा आहे. त्याला ‘इन्व्हर्जन’ किंवा सोप्या भाषेत 'उलटा विचार' म्हणता येईल.  समजा, तुम्हाला तब्येत सुधारायची आहे. तुम्ही सकस आहार घ्याल, व्यायाम कराल, हे स्वाभाविक आहे. पण जर उलटा विचार केला, तर तुम्ही पाहाल की तब्येत बिघडवणाऱ्या सवयी कोणत्या आहेत आणि त्या टाळता येतात का? जंक फूड कमी करता येईल का? व्यसने सोडता येतील का? दिवसभर कॉम्प्युटर किंवा मोबाईलसमोर बसण्याची सवय मोडता येईल का? केवळ या नकारात्मक गोष्टी बंद केल्या, तरी तुमच्या आरोग्यावर सकारात्मक परिणाम होईल. हाच ‘उलट्या विचारा’चा गाभा आहे. समस्येकडे वेगळ्या (उलट्या) दृष्टिकोनातून पाहण्याची संधी देते. चुकांकडे लक्ष देऊन त्या टाळल्या, तर यशस्वी होण्याची शक्यता वाढते.  भारतीय तत्वज्ञानातही याची बीजे सापडतात. सत्याचा शोध घेताना नेति-नेति (हेही नाही, तेही नाही) ह्या पद्धतीने काय टाळायचे ते आधी ठरवले, की अंतिम सत्याचा-वास्तवतेचा मार्ग स्पष्ट होतो.  चार्ली मंगर एकदा म्हणाले होते, "मला फक्त हेच माहित करायचं आहे की मी कुठे मरणार आहे, म्हणजे मी तिथे जाणार नाही." हा विनोदी वाटणारा; पण खोल अर्थ असलेला विचार उलट्या विचारसरणीचे प्रमुख तत्व स्पष्ट करते. अपयश टाळणे, अनेकदा यश मिळवण्याइतकेच महत्त्वाचे असते.  
गुंतवणूक: हुशारीपेक्षा मूर्खपणा टाळा
गुंतवणुकीत यशस्वी होण्यासाठी गहन विश्लेषण आणि गुंतागुंतीच्या तंत्रांचा आधार न घेता, प्रथम मूर्खपणा टाळा. "कुठले शेअर्स मला श्रीमंत करतील?" असा विचार करण्याऐवजी, "कुठल्या चुका माझी संपूर्ण संपत्ती नष्ट करू शकतात?" असा प्रश्न विचारला पाहिजे. अनावश्यक कर्ज घेऊन गुंतवणूक करू नका. सर्वसाधारण बँका देतात त्यापेक्षा एखादी योजना दुप्पट-तिप्पट परतावा देण्याचे वचन देत असेल तर ते कसे खरे मानायचे? अतिशय ‘चमको’ योजनांवर आंधळा विश्वास ठेवू नका. कळपाच्या (मेंढरांच्या) मानसिकतेने व्यवहार करणे, भावनांवर आधारित निर्णय घेणे, आणि अत्याधिक जोखीम पत्करणे यांसारख्या चुका टाळल्या, तरी दीर्घकालीन संपत्ती निर्माण होऊ शकते.  मंगर म्हणतात, "खूप हुशार होण्याचा प्रयत्न करण्यापेक्षा, फक्त सातत्याने मूर्खपणा टाळल्यानेच आम्हाला किती मोठा दीर्घकालीन फायदा झाला आहे हे पाहून आश्चर्य वाटते."
व्यवसाय: चुका टाळा  
एखादी कंपनी यशस्वी होण्यासाठी काय करावे यावर लक्ष केंद्रित करण्याऐवजी, भूतकाळातील अपयशांचे विश्लेषण करून काय करू नये हे समजून घेतले पाहिजे. कोडॅकने, नोकियाने स्मार्टफोनच्या क्रांतीचा अचूक अंदाज लावला नाही आणि आपले वर्चस्व गमावले. "काय आपल्याला बुडवू शकते?" असा प्रश्न विचारल्याने कंपन्या संभाव्य आपत्ती आधीच ओळखून टाळू शकतात. भरमसाठ उत्पादनांच्या मागे न लागता, त्यातील न चालणारी, अद्वितीय नसणारी उत्पादने बंद केलेलीच बारी, नाही का?
निवडीतील गोंधळ कमी करणे  
आजच्या जगात पर्यायांची भरमार आहे. शिक्षण-करियरमध्ये, नोकरी-व्यवसायात, ते अगदी जोडीदार निवडीमध्येही अनेक पर्याय असतात. गोंधळायला होऊ शकते. फारच अचूक-तंतोतंत शोधत बसलो तर काही मिळत नाही आणि काहीही ठरवले नाही तर अतिसामान्य गोष्ट मिळण्याची शक्यता वाढते. याला उपाय म्हणून एक मध्यम-मार्ग म्हणजे, मला ‘काय नको’ हे आधी पक्के ठरवणे, मग उरलेल्या गोष्टींमध्ये निवड करणे. या पद्धतीत निवडीस चांगला वावही राहतो व निराशाही पदरी पडत नाही. ‘उलट विचारांचा’ हा ही एक प्रकार. 
यशासाठी वजाबाकीचा दृष्टिकोन  
आजच्या जगात सतत काहीतरी मिळवण्याची चढाओढ आहे. नवीन कौशल्ये, नव्या रणनीती, अधिक युक्त्या! पण उलट विचारसरणी आपल्याला वजाबाकीचे महत्त्व शिकवते. अनावश्यक गोष्टी ओळखून त्या दूर केल्या, की यशाचा मार्ग आपोआप मोकळा होतो.मग ज्या गोष्टी खरोखरच महत्वाच्या आहेत त्यांच्यावर लक्ष केंद्रित करणे शक्य होते. आजच प्रयत्न करून पहा: "मला अधिक चांगले होण्यासाठी काय करायला हवे?" असा विचार करण्याऐवजी, "मी काय करणे थांबवायला हवे?" असा प्रश्न विचारून पहा. उत्तरं कदाचित आश्चर्यकारक वाटतील, पण ती नक्कीच तुम्हाला योग्य मार्गावर आणतील.

